\section*{Computational Methods}

The (111) metal surfaces were modeled with slabs containing 192 metal atoms arranged in four 8 $\times$ 6 atomic layers. \SI{10}{\angstrom} of vacuum was added in the direction normal to the surface to ensure weak interaction between periodic images of the slab. The size of the slabs in the lateral directions was $20.55 \times 13.35$~\si{\angstrom} for Cu, $22.84 \times 14.82$~\si{\angstrom} for Ag, and $22.47 \times 14.60$~\si{\angstrom} for Au. It is assumed that the presence of the vacancy does not affect the state energetics drastically and, therefore, the vacancy-containing models are used to model pre-existing adatoms. 

DFT calculations were performed using Vienna \emph{ab initio} simulation package (VASP)~\cite{ullmann_131, ullmann_132, ullmann_133, ullmann_134}. The dispersion-corrected~\cite{ullmann_136, ullmann_137} Perdew-Burke-Ernzerhof (PBE) generalized gradient approximation~\cite{ullmann_139} was used as the exchange-correlation functional. 
Spin-polarized electronic states were modeled using a plane wave basis set with the energy cut-off set at \SI{800}{\electronvolt}.
The projector augmented wave formalism was used to describe interactions of atomic cores with valence electrons. The integration over the Brillouin zone was performed using the $3\times 3 \times1$ Monkhorst-Pack $k$-point mesh. 

Atomic positions were optimized until the maximum force on atoms decreased below \SI{0.02}{\electronvolt\per\angstrom}. 
Transition state structures were located using the climbing-image nudged elastic band (NEB) with the VTST code~\cite{ullmann_59}. 
In NEB calculations, an improved initial guess~\cite{ullmann_60, ullmann_99} for the minimum energy path was used and the positions of atoms were relaxed until the maximum force dropped below \SI{0.1}{\electronvolt\per\angstrom}.

%{\comm RZK0804: Energy vs enthalpy.}
